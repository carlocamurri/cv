\documentclass[a4paper, 10pt]{article}

% Lorem ipsum
\usepackage{lipsum}

% Setting font
\usepackage{fontspec}
\setmainfont[
    Path=fonts/open-sans/,
    BoldFont=OpenSans-Bold.ttf,
    ItalicFont=OpenSans-Italic.ttf
]{OpenSans-Regular.ttf}

% Margins
\usepackage[left=1cm, right=1cm, top=0.9cm, bottom=0.9cm]{geometry}

% Remove page numbering
\pagenumbering{gobble}

% Paragraph indentation
\setlength{\parindent}{0pt}

% Section header styling
\usepackage{titlesec}
%% \titleformat{command}[shape]{format}{label}{sep}{before}[after]
\titleformat{\section}[display]{\large\bfseries}{}{0pt}{}[\titlerule]

% Section header spacing
%% \titlespacing*{command}{left}{beforesep}{aftersep}[right]
\titlespacing*{\section}{0pt}{12pt}{3pt}[0pt]

% Tabular for details
\usepackage{array}
\usepackage{tabu}
\newenvironment{details}{\addvspace{3pt} \begin{tabu} to \textwidth {@{}X[2l] X[1r]@{}}}{\end{tabu}}

% Tabular for details specific for work experience modified so that everything fits in one line
\newenvironment{workdetails}{\addvspace{3pt} \begin{tabu} to \textwidth {@{}X[4l] X[1r]@{}}}{\end{tabu}}

% Tabular for initial personal details
\newenvironment{personaldetails}{\begin{tabu} to \textwidth {@{}X[1l] X[1.5c] X[1r]@{}}}{\end{tabu}}

% URL links styling
\usepackage{xcolor}
\usepackage{hyperref}
\definecolor{linkcolor}{rgb}{0, 0.2, 0.6}
\hypersetup{
    colorlinks,
    breaklinks,
    urlcolor=linkcolor,
    linkcolor=linkcolor
}

% Styling for company names
\newcommand{\company}[1]{\normalsize \textbf{#1}}

% Styling for time ranges
\newcommand{\timerange}[1]{\normalsize #1}

% Styling for position title
\newcommand{\position}[1]{\normalsize \textit{#1}}

% Styling for project title
\newcommand{\project}[1]{\normalsize \textbf{#1}}

% Styling for award/certificate title
\newcommand{\award}[1]{\normalsize \textbf{#1}}

% Styling for leadership task
\newcommand{\leadership}[1]{\normalsize \textbf{#1}}

% Styling for skill titles
\newcommand{\skill}[1]{\small \textbf{#1}}

% Bullet point styling for extra information
\usepackage{enumitem}
\newenvironment{info}{\small \begin{itemize}[
    noitemsep,
    topsep=-3pt,
    leftmargin=*,
    align=parleft
]}{\end{itemize}}

% C#
\newcommand{\Csh}{C\#}


\begin{document}
    
    \centerline{\Large \bfseries Carlo Camurri}
    \begin{personaldetails}
        GitHub: \href{https://github.com/carlocamurri}{carlocamurri} & & \href{mailto:carlo.camurri98@gmail.com}{carlo.camurri98@gmail.com} \\
        LinkedIn: \href{https://www.linkedin.com/in/carlo-camurri/}{carlo-camurri} & Bilingual Italian/English; French (B2 level) & +44 (0)7729751799
    \end{personaldetails}

    \section*{Education}

        \begin{details} 
            \company{University College London (UCL)} & \timerange{September 2016 -- Present}
        \end{details}

        \begin{info}
            \item Master of Engineering in Computer Science (3rd Year)
            \item Year 1: first class, 81\% average (top 10-percentile); Year 2: first class, 85\% average; Year 3: first class, 87\% average
        \end{info}

        \begin{details}
            \company{The International School in Genoa} & \timerange{September 2014 -- May 2016}
        \end{details}

        \begin{info}
            \item International Baccaulaureate (IB) 39/45 overall; 7 (A*) in Higher Level Physics
            \item Extended Essay (Physics): ``Projectile motion of small extended objects''; using \href{http://www.infomus.org/eyesweb_ita.php}{EyesWeb} -- software for real time multimodal analysis
        \end{info}

    \section*{Work Experience}
    
        \begin{workdetails}
            \company{G-Research}, \position{Software Engineering Intern} & \timerange{June 2019 -- Present}
        \end{workdetails}
    
        \begin{info}
            \item Worked on several applications for the management and scheduling of jobs in large clusters of Windows and Linux based nodes
            \item Developed a new application to provide researchers with an interface to monitor and cancel their jobs
            \item Technologies: .NET Core (\Csh{}), Angular (TypeScript), SQL, \href{https://research.cs.wisc.edu/htcondor/}{HTCondor}, Docker, Kubernetes, Jenkins
        \end{info}
    
        \begin{workdetails}
            \company{NeuroResponse, NHS Innovation Accelerator}, \position{Full Stack Software Engineer} & \timerange{July 2018 -- June 2019}
        \end{workdetails}

        \begin{info}
            \item Developed full stack application to enable patients suffering from multiple sclerosis to record their symptoms and fill in questionnaires pertaining to their courses of medication
            \item Comprised of a REST API, a mobile application for the patients and a web portal for the clinicians
            \item Technologies: Python, AWS (Lambda, DynamoDB, Cognito), Android (Kotlin), React
        \end{info}

        \begin{workdetails}
            \company{Casa Paganini -- InfoMus Research Center, DIBRIS, University of Genoa}, \position{Research Intern} & \timerange{June -- July 2017}
        \end{workdetails}

        \begin{info}
            \item Applied machine learning techniques to the analysis of a motion capture repository of human movement, such as random forests and hierarchical clustering; supervised by Prof. Maurizio Mancini
            \item Technologies: Python (Pandas, Numpy, Scikit-Learn), \href{http://www.qualisys.com/software/qualisys-track-manager/}{Qualisys motion capture system}
        \end{info}

    \section*{Projects}

        \begin{details}
            \project{NotiPlex, an Android multi-device notification framework} & \timerange{October 2017 -- March 2018}
        \end{details}
        
        \begin{info}
            \item In collaboration with Microsoft, Inria, UCL
            \item Team Leader of development team tasked with developing a notification management application
            \item The application enables users to create configurations specifying which device should receive incoming notifications based on time and location
            \item Technologies: Android (Java), Firebase, Node.js
        \end{info}

        \begin{details}
            \project{VR Health Living} & \timerange{January -- May 2017}
        \end{details}
        
        \begin{info}
            \item In collaboration with Imperial College Healthcare NHS Trust and UCL
            \item Developed a virtual reality dancing application to encourage children between the ages of 8-12 to do physical activity within a context of gamification
            \item The application was designed to be portable and was deployed with support for Samsung VR
            \item Technologies: C\#, Unity, Samsung VR
        \end{info}

    \section*{Awards and Certificates}

        \begin{details}
            \award{LearnHack 4.0 Hackathon} & \timerange{November 2017}
        \end{details}
        
        \begin{info}
            \item LabNotes Challenge Winner
            \item Created lab notebook to help researchers log their data and notes for future use in research papers using React and Node.js 
        \end{info}
        
        \begin{details}
            \award{Bank of America Trading Application Hackathon} & \timerange{November 2017}
        \end{details}
        
        \begin{info}
            \item Third place
            \item Developed front-end and back-end for a trading application using Google Trends and Aylien News APIs
        \end{info}

        \begin{details}
            \award{Machine Learning course by Andrew Ng, Stanford University Online} & \timerange{December 2016 -- March 2017}
        \end{details}

        \begin{info}
            \item Topics covered: Linear and logistic regression, regularization, neural networks, SVM, PCA, k-means clustering, anomaly detection, recommender systems
        \end{info}

    \section*{Leadership and Service}

        \begin{details}
            \leadership{Outreach Co-Director, Students' Union UCL Technology Society} & \timerange{January 2018 -- June 2018}
        \end{details}

        \begin{info}
            \item Actively contributed in organizing software engineering-related talks and events involving external companies such as Next Jump and KPMG. 
        \end{info}

        \begin{details}
            \leadership{Hackstart: Teaching basic Machine Learning} & \timerange{October 2017}
        \end{details}

        \begin{info}
            \item Hosted interactive workshop using Python and Jupyter to teach Numpy and basic unsupervised machine learning techniques (K-means algorithm) to both high school and university students with non-technical backgrounds
        \end{info}

    \section*{Skills}
        
        \skill{Programming Languages}: 
        \small{Proficient in Python, Java, C, JavaScript; Familiar with Kotlin, C\#, C++, TypeScript, Haskell, Matlab}

        \skill{Web Development}: 
        \small{React, Node.js, HTML, CSS}

        \skill{Frameworks and Tools}:
        \small{Android, TensorFlow, Unity, EyesWeb, Git, Docker}

        \skill{Software Libraries}:
        \small{Keras, Scikit-learn, Numpy, Pandas}

        \skill{Operating Systems}: 
        \small{GNU/Linux (Ubuntu, Debian), Android, Microsoft Windows}

    \end{document}
